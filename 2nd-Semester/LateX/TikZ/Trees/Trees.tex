\documentclass{article}
\usepackage{forest}
\usepackage{tikz}
\title{Drawing Tree with Forest}
\author{Credit: https://github.com/HasibLearntogrow/LatexPractice}
\begin{document}
	\maketitle
	\tableofcontents
	\clearpage
	
	\section{Create Node and Color:}
		\subsection{Use Various Shapes:}
		\subsubsection{Normal Node:}
		\begin{forest}
			[A];
		\end{forest}
		
		\subsubsection{Circle Shaped Node:}
		\begin{forest}
			[A,circle,draw]
		\end{forest}
		
		\subsubsection{Rectangle Shaped Node}
		\begin{forest}
			[A,rectangle,draw]
		\end{forest}	
		
		\subsubsection{All shape in one tress:}
		\begin{forest}
			[A,rectangle,draw[B,rectangle,draw][C]]
		\end{forest}
		
		\subsection{Coloring of node:}
		\begin{forest}
			[A,circle,draw,color=black,red,fill=red!20]
		\end{forest}
		
%-------------------------------------------------------------------------------------------------

	\section{Simple Tree:}
		\subsection{Parent Node: A and Child Node B,C:}
		% Procedure: [p[c1][c2].......]
		\begin{forest}
			[A[B[D]][C]]
		\end{forest}
		
	\subsection{More node with tree :}
     \begin{forest}
     	[A[B[D[1][2]][E[3][4]]][C[F[5][6]][G[7][8]]]]
     \end{forest}
     
      \subsection{More node with tree with a single shape(Circle):}
      \begin{forest} for tree={circle,draw}
      	[A[B[D[1][2]][E[3][4]]][C[F[5][6]][G[7][8]]]]
      \end{forest}
      
      \subsection{More node with tree with a single shape(Rectangle):}
      \begin{forest} for tree={rectangle,draw}
      	[A[B[D[1][2]][E[3][4]]][C[F[5][6]][G[7][8]]]]
      \end{forest}
      
      \subsection{Draw Horizontal tree:}
      \begin{forest} for tree={grow=0}
      	[A[B[D[1][2]][E[3][4]]][C[F[5][6]][G[7][8]]]]
      \end{forest}
      
      \subsection{Edge Label:}
      \begin{forest}
      	[A[B, edge label={node[midway, left] {Label 1}}[D,inner sep=2pt[1][2]][E[3][4]]][C[F[5][6]][G[7][8]]]]
      \end{forest}
      
%-------------------------------------------------------------------------------------------------
	\section{Shifting of Tree}
		\subsection{left side shift of Tree(calign=first):}
		\begin{forest}
			[A,calign=first[B[1][2]][c[3][4]]]
		\end{forest}
		
		\subsection{Right side shift of Tree(calign=last):}
		\begin{forest}
			[A,calign=last[B[1][2]][C[3][4]]]
		\end{forest}
		
		\subsection{Align all first child node connected with immediate parent node by straight line:}
		\begin{forest} for tree={calign=first}
			[A[B[D[1][2]][E[3][4]]][C[F[5][6]][G[7][8]]]]
		\end{forest}
		
%-------------------------------------------------------------------------------------------------
	\section{Practice:}
		\subsection{Practice 01:}
		\begin{tikzpicture}	
			\node(X) at (0,0) {};
			\node(X) at (-2,0) {X};
			\node(X) at (-2,1) {X};
		\end{tikzpicture}
		\begin{forest}
			[A,circle,draw
				[B]
				[C, name=Cnode]
			]
		\end{forest}
		\begin{tikzpicture}	
			\node(X) at (0,0) {};
			\node(X) at (2,0) {X};
			\node(1) at (2,1) {X};
			\draw[->] (Cnode) to (1);
		\end{tikzpicture}
		
		
		
\end{document}
