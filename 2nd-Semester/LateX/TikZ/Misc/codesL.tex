\documentclass{article}
\usepackage{listings}
\usepackage{xcolor}
\usepackage[margin=1in]{geometry}

\lstset{
    basicstyle=\ttfamily\footnotesize,
    keywordstyle=\color{blue},
    stringstyle=\color{red},
    commentstyle=\color{green!60!black},
    numbers=left,
    numberstyle=\tiny\color{gray},
    stepnumber=1,
    numbersep=10pt,
    frame=single,
    showstringspaces=false,
    breaklines=true,
    captionpos=b
}

\begin{document}

\section*{Cosine Similarity Code}
\begin{lstlisting}[language=Java, caption=Cosine Similarity Implementation]
#CosineSimilarity

import java.util.Scanner;
import java.util.HashMap;
import java.util.Map;

class similarityEstimation {
    Map<String, Integer> wordFreqEstimation(String str) {
        Map<String, Integer> freqMap = new HashMap<String, Integer>();
        Integer count = null;

        String delim = " ";
        String[] token = str.split(delim);
        String word;

        for (int i = 0; i < token.length; i++) {
            word = token[i];
            count = freqMap.get(word);

            if (count == null) {
                count = 1;
            } else {
                count++;
            }
            freqMap.put(word, count);
        }
        return freqMap;
    }

    Double cosSim(String str1, String str2) {
        Map<String, Integer> docfreq1 = wordFreqEstimation(str1);
        Map<String, Integer> docfreq2 = wordFreqEstimation(str2);
        Double cosine_similarity;
        double mul = 0.0f;
        double fr1 = 0.0f;
        double fr2 = 0.0f;

        for (String key1 : docfreq1.keySet()) {
            fr1 += Math.pow(docfreq1.get(key1), 2);
        }
        for (String key2 : docfreq2.keySet()) {
            fr2 += Math.pow(docfreq2.get(key2), 2);
        }

        for (String key1 : docfreq1.keySet()) {
            if (docfreq2.containsKey(key1)) {
                mul += docfreq1.get(key1) * docfreq2.get(key1);
            }
        }
        cosine_similarity = (Double) (mul / Math.sqrt(fr1 * fr2));
        return cosine_similarity;
    }
}

public class cosineSimilarity {
    public static void main(String[] args) {
        similarityEstimation similarityEstimationObj = new similarityEstimation();
        Map<String, Integer> fMap = similarityEstimationObj.wordFreqEstimation(
            "the best data science course in the best university");
        Map<String, Integer> fMap2 = similarityEstimationObj.wordFreqEstimation(
            "the best course in university of science");
        System.out.println(fMap);
        System.out.println(fMap2);
        Double x = similarityEstimationObj.cosSim(
            "the best data science course in the best university",
            "the best course in university of science");
        System.out.println(x);
    }
}
\end{lstlisting}

\section*{Jaccard Similarity Code}
\begin{lstlisting}[language=Java, caption=Jaccard Similarity Implementation]
#JaccardSimilarity

import java.util.HashMap;
import java.util.HashSet;
import java.util.Map;
import java.util.Scanner;
import java.util.Set;

public class jaccardSimilarity {

    class wordFrequency {
        Map<String, Integer> calculate(String str) {
            Map<String, Integer> freqMap = new HashMap<String, Integer>();
            String[] token1 = str.split("\\s+");
            String word;
            Integer count = null;
            for (int i = 0; i < token1.length; i++) {
                word = token1[i];
                count = freqMap.get(word);

                if (count == null) {
                    count = 1;
                } else {
                    count++;
                }
                freqMap.put(word, count);
            }
            return freqMap;
        }
    }

    class jacSim {
        Map<String, Integer> freqMap1 = new HashMap<String, Integer>();
        Map<String, Integer> freqMap2 = new HashMap<String, Integer>();

        jacSim(String str1, String str2) {
            wordFrequency wFObj = new wordFrequency();
            freqMap1 = wFObj.calculate(str1);
            freqMap2 = wFObj.calculate(str2);
        }

        double calculate() {
            Set<String> set1 = new HashSet<String>();
            Set<String> set2 = new HashSet<String>();
            for (String key1 : freqMap1.keySet()) {
\end{lstlisting}

\section*{SMC}
\begin{lstlisting}[language=Java, caption=Cosine Similarity Implementation]
#SMC
import java.util.Scanner;
public class SMC {
	public static void main(String[] args) {
		Scanner myInput = new Scanner(System.in);
		String x = myInput.nextLine();
		String y = myInput.nextLine();
		int f01=0, f10=0, f00=0, f11=0;
		double SMC=0;
		for(int i = 0; i<x.length(); i++) {
			if(x.charAt(i) == '0' && y.charAt(i) == '1') {
				f01++;
			}
			if(x.charAt(i) == '1' && y.charAt(i) == '0') {
				f10++;
			}
			if(x.charAt(i) == '0' && y.charAt(i) == '0') {
				f00++;
			}
			if(x.charAt(i) == '1' && y.charAt(i) == '1') {
				f11++;
			}
		}
		SMC = (double)(f11+f00)*(f01+f10+f11+f00)/100;
		System.out.println(SMC);
	}
}


\end{lstlisting}

\section*{FileProblem}
\begin{lstlisting}[language=Java, caption=Cosine Similarity Implementation]
import java.io.*;
import java.util.HashMap;
import java.util.Map;
import java.util.Scanner;
public class FileProblem {	
	public static void main(String[] args) {
		String filePath1 = "D:\\Academic-Coding\\2nd-Semester\\OOP\\eclipse-workplace\\LabExamPractice\\src\\oldmast.txt";
		String filePath2 = "D:\\Academic-Coding\\2nd-Semester\\OOP\\eclipse-workplace\\LabExamPractice\\src\\trans.txt";
		FileProblem f = new FileProblem();
		FileMatch fileMatchObj = f.new FileMatch(filePath1, filePath2);
		fileMatchObj.read();
		String outPath = "D:\\Academic-Coding\\2nd-Semester\\OOP\\eclipse-workplace\\LabExamPractice\\src\\newmast.txt";
		String errorPath = "D:\\Academic-Coding\\2nd-Semester\\OOP\\eclipse-workplace\\LabExamPractice\\src\\log.txt";;
		fileMatchObj.newRecord(outPath, errorPath);
	}
	class FileMatch{
		String filePath1, filePath2;
		Map<String, Double> map = new HashMap<String, Double>();
		Map<String, Double> map2 =  new HashMap<String, Double>();
		FileMatch(String filePath1, String filePath2){
			this.filePath1 = filePath1;
			this.filePath2 = filePath2;
			this.map = map;
			this.map2 = map2;
		}
		void read(){
			System.out.println("oldmast.txt");
			try(BufferedReader reader = new BufferedReader(new FileReader(filePath1))){
				String line;
				while((line = reader.readLine())!= null) {
					System.out.println(line);
				}
			}
			catch(IOException e){
				e.printStackTrace();
			}
			System.out.println("trans.txt");
			try(BufferedReader reader = new BufferedReader(new FileReader(filePath2))){
				String line;
				while((line = reader.readLine())!= null) {
					System.out.println(line);
				}
			}
			catch(IOException e){
				e.printStackTrace();
			}
		}
		void newRecord(String outPath, String errorPath){
			map = new HashMap<String, Double>();
			try(BufferedReader reader = new BufferedReader(new FileReader(filePath1))){
				String line;
				while((line = reader.readLine())!= null) {
					String[] words;
					words = line.split("\\s+");
					map.put(words[0], Double.parseDouble(words[2]));	
				}
			}
			catch(IOException e){
				e.printStackTrace();
			}
			map2 = new HashMap<String, Double>();
			Map<String, Double> mapOut = new HashMap<String, Double>();
			Map<String, Double> mapLog = new HashMap<String, Double>();
			try(BufferedReader reader = new BufferedReader(new FileReader(filePath2))){
				String line;
				while((line = reader.readLine())!= null) {
					String[] words;
					words = line.split("\\s+");
					if(!map2.containsKey(words[0])) {
						map2.put(words[0], Double.parseDouble(words[1]));
					}
					else {
						map2.put(words[0], map2.get(words[0])+Double.parseDouble(words[1]));
					}
				}
			}
			catch(IOException e){
				e.printStackTrace();
			}
			for(String id: map2.keySet()) {
				if(map.containsKey(id)) {
					Double balance1 = map.get(id);
					Double balance2 = map2.get(id);
					Double balance = balance1 + balance2;
					mapOut.put(id,balance);
				}
				else if(!map.containsKey(id)) {
					mapLog.put(id, map2.get(id));
				}
			}
			for(String id: map.keySet()) {
				if(!map2.containsKey(id)) {
					mapOut.put(id, map.get(id));
				}
			}
			try(BufferedWriter br = new BufferedWriter(new FileWriter(outPath))){
				for(String id: mapOut.keySet()) {
					br.write(id+" "+mapOut.get(id));
					br.newLine();
				}
			}
			catch(IOException e){
				e.printStackTrace();
			}
			try(BufferedWriter br = new BufferedWriter(new FileWriter(errorPath))){
				for(String id: mapLog.keySet()) {
					br.write("Unmatched transaction for A/C no. "+id);
					br.newLine();
				}
			}
			catch(IOException e){
				e.printStackTrace();
			}
		}
	}
}
\end{lstlisting}

\section*{V1OOP:One(Course&Students)}
\begin{lstlisting}[language=Java, caption=Cosine Similarity Implementation]
package V1OOPProgrammingLab;
import java.io.*;
import java.util.*;
public class OneV1 {
	class course{
		String courseName;
		String[] stdIDs;
		Map<String, String[]> courseMap = new HashMap<String, String[]>();
		course() {
			try(BufferedReader reader = new BufferedReader(new FileReader("D:\\Academic-Coding\\2nd-Semester\\OOP\\eclipse-workplace\\LabExamPractice\\src\\V1OOPProgrammingLab\\course.txt"))){
				String line;
				int i = 0;
				while((line=reader.readLine())!= null) {
					String[] elements = line.split("\t");
					this.courseName = elements[0];
					this.stdIDs = elements[1].split(", ");
					i++;
				}
				courseMap.put(courseName, stdIDs);
			}
			catch(IOException e){
				e.printStackTrace();
			}
		}
		void readStds(String courseName) {
			String[] stds = courseMap.get(courseName);
			for(int i = 0; i < stds.length; i++) {
				System.out.println(stds[i]);
			}
		}
		void readAll() {
			for(String cN: courseMap.keySet()) {
				String[] stds = courseMap.get(cN);
				System.out.print(cN + " : ");
				for(int i = 0; i < stds.length; i++) {
					System.out.print(stds[i]);
					if(i < stds.length - 1) {
						System.out.print(", ");
					}
				}
			}
		}
	}	
	public static void main(String[] args) {
		OneV1 oneObj = new OneV1();
		course cObj = oneObj.new course();
		cObj.readStds("CSE312");
		cObj.readAll();
	}
}

\end{lstlisting}

\section*{Cosine Similarity Code}
\begin{lstlisting}[language=Java, caption=Cosine Similarity Implementation]

\end{lstlisting}\\
\\
\section*{Cosine Similarity Code}
\begin{lstlisting}[language=Java, caption=Cosine Similarity Implementation]

\end{lstlisting}\\
\\
\section*{Cosine Similarity Code}
\begin{lstlisting}[language=Java, caption=Cosine Similarity Implementation]

\end{lstlisting}\\
\\
\section*{Cosine Similarity Code}
\begin{lstlisting}[language=Java, caption=Cosine Similarity Implementation]

\end{lstlisting}\\
\\
\\
\section*{Cosine Similarity Code}
\begin{lstlisting}[language=Java, caption=Cosine Similarity Implementation]

\end{lstlisting}\\
\\
\section*{Cosine Similarity Code}
\begin{lstlisting}[language=Java, caption=Cosine Similarity Implementation]

\end{lstlisting}\\
\\
\section*{Cosine Similarity Code}
\begin{lstlisting}[language=Java, caption=Cosine Similarity Implementation]

\end{lstlisting}\\
\section*{Cosine Similarity Code}
\begin{lstlisting}[language=Java, caption=Cosine Similarity Implementation]

\end{lstlisting}\\
\section*{Cosine Similarity Code}
\begin{lstlisting}[language=Java, caption=Cosine Similarity Implementation]

\end{lstlisting}\\
\section*{Cosine Similarity Code}
\begin{lstlisting}[language=Java, caption=Cosine Similarity Implementation]

\end{lstlisting}\\
\section*{Cosine Similarity Code}
\begin{lstlisting}[language=Java, caption=Cosine Similarity Implementation]

\end{lstlisting}\\
\section*{Cosine Similarity Code}
\begin{lstlisting}[language=Java, caption=Cosine Similarity Implementation]

\end{lstlisting}\\
\section*{Cosine Similarity Code}
\begin{lstlisting}[language=Java, caption=Cosine Similarity Implementation]

\end{lstlisting}\\
\section*{Cosine Similarity Code}
\begin{lstlisting}[language=Java, caption=Cosine Similarity Implementation]

\end{lstlisting}\\
\section*{Cosine Similarity Code}
\begin{lstlisting}[language=Java, caption=Cosine Similarity Implementation]

\end{lstlisting}\\

\end{document}
