\documentclass[11pt]{article}
\usepackage{coling2020} %The coling2020 package is specific to the COLING 2020 conference (International Conference on Computational Linguistics). It defines the formatting and style guidelines required for papers submitted to that conference.
\usepackage{times} %font
\usepackage{pdfpages} %needed for hex color
\usepackage{color} %font color
\usepackage[misc]{ifsym} %for letter emoji
\usepackage{amssymb} %for symbols like \blacktriangleright
\usepackage{enumerate} %for enumerating
\usepackage[shortlabels]{enumitem} %for labels
\usepackage[cmex10]{amsmath} %for \begin{equation*}
\usepackage{amsmath,amssymb,amsfonts} %for \begin{equation*}
\graphicspath{{Figures/EPS/}{figures/}} %for \begin{split} in eqn
\usepackage[usestackEOL]{stackengine} %for \begin{split} in eqn
\usepackage{graphicx} %for figures
\graphicspath{{Figures/EPS/}{figures/}} %for figure path so that we don't have to specify the path
\usepackage{tabulary} %for picture side by side
\usepackage{float} %for using the [H] option in pie figure


\colingfinalcopy %The \colingfinalcopy command is a specific command provided by the coling2020 package to indicate that the document is in its final version, ready for submission. This command typically: Applies Final Formatting, Suppresses Draft-Only Features

\newcommand*{\affaddr}[1]{#1} % No op here. Customize it for different styles.

\DeclareMathOperator*{\argmin}{\arg\!\min} 
\DeclareMathOperator*{\argmax}{\arg\!\max}

\title{STA 251 Syllabus \\}

\author{Sumaiya Tabassum\\
 Department of Computer Science and Engineering\\
 \affaddr{University of Chittagong, Chittagong, Bangladesh}\\
}


\begin{document}
\maketitle
\pagestyle{plain}

\section{Data Visualization}
\begin{itemize}
\item Histogram / Stack Graph
\item Line Chart (only mentioneded in class)
\item Pie Chart (only mentioneded in class)
\end{itemize}

\section{Counting Principles(see notes)}
\begin{itemize}
\item Product Rule
\item Sum Rule
\item Division Rule
\item \emph{Related Topics:}
\begin{itemize}
\item Vector Space
\item Function Space
\item Balanced Parentheses Problem
\end{itemize}
\end{itemize}

\section{Set Theory}
\begin{itemize}
\item Ordered Sets
\item Unordered Sets
\end{itemize}

\section{Functions}
\begin{itemize}
\item One-to-One Functions
\item Onto Functions
\item Bijection
\end{itemize}

\section{Relations}
\begin{itemize}
\item Relations (General overview)
\end{itemize}
\nopagebreak
\section{Descriptive Statistics}
\begin{itemize}
\item \emph{Measures of Centrality:}
\begin{itemize}
\item Mean / Average
\item Median
\item Distribution
\end{itemize}
\item \emph{Types of Means:}
\begin{itemize}
\item Arithmetic Mean
\item Geometric Mean
\end{itemize}
\item \emph{Variability \& Dispersion:}
\begin{itemize}
\item Variance
\item Standard Deviation
\end{itemize}
\item Noisy Data
\end{itemize} 

\section{Probability}
\begin{itemize}
\item Sample Space and Events
\item Proves of Probability
\item \emph{Probability Axioms:}
\begin{itemize}
\item Non-Negativity
\item Countable Additivity
\item Normalization
\end{itemize}
\item Bayes’ Law
\item \emph{Independence:}
\begin{itemize}
\item Mutually Independent
\item Pairwise Independent
\item K-way Independent
\end{itemize}
\item \emph{Random Variables:}
\begin{itemize}
\item Binomial Random Variable
\item Geometric Random Variable
\item Probability Mass Function (PMF)
\item Cumulative Distribution Function (CDF)
\end{itemize}
\item Conditional Probability
\item Law of total probability
\item Expectation 
\item Conditional Expectation
\item Law of Total Probability
\item Linearity of Expectation
\item Variance and Covariance
\item Covariance from Scatter Plot
\pagebreak
\item \emph{Example Problems:}
\begin{itemize}
\item Monty Hall Problem
\item Birthday Problem Paradox
\item Two-Envelope Problem
\item Light Bulb Problem
\item Prisoner Problem
\item C Program Delay Problem
\item Hat Check Problem
\item Coupon Collector's Problem
\end{itemize}
\end{itemize} 

\end{document}